%************************************************
\chapter{Eigentlicher Text}
\label{chp:text}
%************************************************

\quotes{
	``Die Suppe wird nicht so hei\ss{} gegessen, wie sie gekocht wird.''
}{Nilpferd}
\\

Gleich das Wichtigste vorweg: Es handelt sich hier nicht um eine
Einkaufst\"ute oder eine T\"ute zum Rauchen oder so, sondern um eine
T\"ute f\"ur Suppenzutaten, eine T\"utensuppe sozusagen. Nur halt ohne
die Suppe, eine T\"ute also.

Jedenfalls befand sich die Suppe oder besser geschrieben die
Suppenzutaten, hier aber der Einfachheit halber nur: die Suppe, in der T\"ute.\\
Und wollte heraus, denn was soll die Suppe so alleine in der T\"ute? Eben.\\
Die Suppe stapelte sich also in der T\"ute, und konnte dadurch aus ihr
herausspringen.

Dann spazierte sie in der K\"uche umher, und versuchte, einen Sinn in
ihrem Dasein zu finden. Oder besser: eine M\"oglichkeit, wie aus den
trockenen Nudeln und den staubigen Gew\"urzen eine schmackhafte und
wohltuende Suppe werden kann.

Die Gelegenheit war nicht weit entfernt, denn auf dem Herd befand sich
schon ein Topf mit kochendem Wasser. \textit{Wo kam das denn her?}\\
Egal, die Suppe war damit zufrieden, stellte den Herd vorsorglich
etwas herunter - sie wollte ja nicht \"uberkochen - und sprang in das
hei\ss{}e Bad. F\"ur die Suppe war das zum Gl\"uck nicht unangenehm,
sondern ganz im Gegenteil: angenehm!

Nach etwa 10 Minuten waren die Nudeln al dente, und die staubigen
Gew\"urze hatten sich gut mit dem Wasser angefreundet bzw. vermischt.
Die Suppe, die aus der T\"ute sprang, war damit verzehrfertig und
h\"upfte in den Teller auf dem K\"uchentisch.\\
Zum Gl\"uck sa\ss{} in der K\"uche schon ein Nilpferd, welches die Suppe
genussvoll verspeiste.

Ende.
